% Options for packages loaded elsewhere
\PassOptionsToPackage{unicode}{hyperref}
\PassOptionsToPackage{hyphens}{url}
%
\documentclass[
  ignorenonframetext,
  aspectratio=32,
]{beamer}
\usepackage{pgfpages}
\setbeamertemplate{caption}[numbered]
\setbeamertemplate{caption label separator}{: }
\setbeamercolor{caption name}{fg=normal text.fg}
\beamertemplatenavigationsymbolshorizontal
% Prevent slide breaks in the middle of a paragraph
\widowpenalties 1 10000
\raggedbottom
\setbeamertemplate{part page}{
  \centering
  \begin{beamercolorbox}[sep=16pt,center]{part title}
    \usebeamerfont{part title}\insertpart\par
  \end{beamercolorbox}
}
\setbeamertemplate{section page}{
  \centering
  \begin{beamercolorbox}[sep=12pt,center]{part title}
    \usebeamerfont{section title}\insertsection\par
  \end{beamercolorbox}
}
\setbeamertemplate{subsection page}{
  \centering
  \begin{beamercolorbox}[sep=8pt,center]{part title}
    \usebeamerfont{subsection title}\insertsubsection\par
  \end{beamercolorbox}
}
\AtBeginPart{
  \frame{\partpage}
}
\AtBeginSection{
  \ifbibliography
  \else
    \frame{\sectionpage}
  \fi
}
\AtBeginSubsection{
  \frame{\subsectionpage}
}

\usepackage{amsmath,amssymb}
\usepackage{iftex}
\ifPDFTeX
  \usepackage[T1]{fontenc}
  \usepackage[utf8]{inputenc}
  \usepackage{textcomp} % provide euro and other symbols
\else % if luatex or xetex
  \usepackage{unicode-math}
  \defaultfontfeatures{Scale=MatchLowercase}
  \defaultfontfeatures[\rmfamily]{Ligatures=TeX,Scale=1}
\fi
\usepackage{lmodern}
\usetheme[]{AnnArbor}
\usecolortheme{lily}
\ifPDFTeX\else  
    % xetex/luatex font selection
\fi
% Use upquote if available, for straight quotes in verbatim environments
\IfFileExists{upquote.sty}{\usepackage{upquote}}{}
\IfFileExists{microtype.sty}{% use microtype if available
  \usepackage[]{microtype}
  \UseMicrotypeSet[protrusion]{basicmath} % disable protrusion for tt fonts
}{}
\makeatletter
\@ifundefined{KOMAClassName}{% if non-KOMA class
  \IfFileExists{parskip.sty}{%
    \usepackage{parskip}
  }{% else
    \setlength{\parindent}{0pt}
    \setlength{\parskip}{6pt plus 2pt minus 1pt}}
}{% if KOMA class
  \KOMAoptions{parskip=half}}
\makeatother
\usepackage{xcolor}
\newif\ifbibliography
\setlength{\emergencystretch}{3em} % prevent overfull lines
\setcounter{secnumdepth}{-\maxdimen} % remove section numbering

\usepackage{color}
\usepackage{fancyvrb}
\newcommand{\VerbBar}{|}
\newcommand{\VERB}{\Verb[commandchars=\\\{\}]}
\DefineVerbatimEnvironment{Highlighting}{Verbatim}{commandchars=\\\{\}}
% Add ',fontsize=\small' for more characters per line
\usepackage{framed}
\definecolor{shadecolor}{RGB}{241,243,245}
\newenvironment{Shaded}{\begin{snugshade}}{\end{snugshade}}
\newcommand{\AlertTok}[1]{\textcolor[rgb]{0.68,0.00,0.00}{#1}}
\newcommand{\AnnotationTok}[1]{\textcolor[rgb]{0.37,0.37,0.37}{#1}}
\newcommand{\AttributeTok}[1]{\textcolor[rgb]{0.40,0.45,0.13}{#1}}
\newcommand{\BaseNTok}[1]{\textcolor[rgb]{0.68,0.00,0.00}{#1}}
\newcommand{\BuiltInTok}[1]{\textcolor[rgb]{0.00,0.23,0.31}{#1}}
\newcommand{\CharTok}[1]{\textcolor[rgb]{0.13,0.47,0.30}{#1}}
\newcommand{\CommentTok}[1]{\textcolor[rgb]{0.37,0.37,0.37}{#1}}
\newcommand{\CommentVarTok}[1]{\textcolor[rgb]{0.37,0.37,0.37}{\textit{#1}}}
\newcommand{\ConstantTok}[1]{\textcolor[rgb]{0.56,0.35,0.01}{#1}}
\newcommand{\ControlFlowTok}[1]{\textcolor[rgb]{0.00,0.23,0.31}{#1}}
\newcommand{\DataTypeTok}[1]{\textcolor[rgb]{0.68,0.00,0.00}{#1}}
\newcommand{\DecValTok}[1]{\textcolor[rgb]{0.68,0.00,0.00}{#1}}
\newcommand{\DocumentationTok}[1]{\textcolor[rgb]{0.37,0.37,0.37}{\textit{#1}}}
\newcommand{\ErrorTok}[1]{\textcolor[rgb]{0.68,0.00,0.00}{#1}}
\newcommand{\ExtensionTok}[1]{\textcolor[rgb]{0.00,0.23,0.31}{#1}}
\newcommand{\FloatTok}[1]{\textcolor[rgb]{0.68,0.00,0.00}{#1}}
\newcommand{\FunctionTok}[1]{\textcolor[rgb]{0.28,0.35,0.67}{#1}}
\newcommand{\ImportTok}[1]{\textcolor[rgb]{0.00,0.46,0.62}{#1}}
\newcommand{\InformationTok}[1]{\textcolor[rgb]{0.37,0.37,0.37}{#1}}
\newcommand{\KeywordTok}[1]{\textcolor[rgb]{0.00,0.23,0.31}{#1}}
\newcommand{\NormalTok}[1]{\textcolor[rgb]{0.00,0.23,0.31}{#1}}
\newcommand{\OperatorTok}[1]{\textcolor[rgb]{0.37,0.37,0.37}{#1}}
\newcommand{\OtherTok}[1]{\textcolor[rgb]{0.00,0.23,0.31}{#1}}
\newcommand{\PreprocessorTok}[1]{\textcolor[rgb]{0.68,0.00,0.00}{#1}}
\newcommand{\RegionMarkerTok}[1]{\textcolor[rgb]{0.00,0.23,0.31}{#1}}
\newcommand{\SpecialCharTok}[1]{\textcolor[rgb]{0.37,0.37,0.37}{#1}}
\newcommand{\SpecialStringTok}[1]{\textcolor[rgb]{0.13,0.47,0.30}{#1}}
\newcommand{\StringTok}[1]{\textcolor[rgb]{0.13,0.47,0.30}{#1}}
\newcommand{\VariableTok}[1]{\textcolor[rgb]{0.07,0.07,0.07}{#1}}
\newcommand{\VerbatimStringTok}[1]{\textcolor[rgb]{0.13,0.47,0.30}{#1}}
\newcommand{\WarningTok}[1]{\textcolor[rgb]{0.37,0.37,0.37}{\textit{#1}}}

\providecommand{\tightlist}{%
  \setlength{\itemsep}{0pt}\setlength{\parskip}{0pt}}\usepackage{longtable,booktabs,array}
\usepackage{calc} % for calculating minipage widths
\usepackage{caption}
% Make caption package work with longtable
\makeatletter
\def\fnum@table{\tablename~\thetable}
\makeatother
\usepackage{graphicx}
\makeatletter
\def\maxwidth{\ifdim\Gin@nat@width>\linewidth\linewidth\else\Gin@nat@width\fi}
\def\maxheight{\ifdim\Gin@nat@height>\textheight\textheight\else\Gin@nat@height\fi}
\makeatother
% Scale images if necessary, so that they will not overflow the page
% margins by default, and it is still possible to overwrite the defaults
% using explicit options in \includegraphics[width, height, ...]{}
\setkeys{Gin}{width=\maxwidth,height=\maxheight,keepaspectratio}
% Set default figure placement to htbp
\makeatletter
\def\fps@figure{htbp}
\makeatother

\makeatletter
\makeatother
\makeatletter
\makeatother
\makeatletter
\@ifpackageloaded{caption}{}{\usepackage{caption}}
\AtBeginDocument{%
\ifdefined\contentsname
  \renewcommand*\contentsname{Table of contents}
\else
  \newcommand\contentsname{Table of contents}
\fi
\ifdefined\listfigurename
  \renewcommand*\listfigurename{List of Figures}
\else
  \newcommand\listfigurename{List of Figures}
\fi
\ifdefined\listtablename
  \renewcommand*\listtablename{List of Tables}
\else
  \newcommand\listtablename{List of Tables}
\fi
\ifdefined\figurename
  \renewcommand*\figurename{Figure}
\else
  \newcommand\figurename{Figure}
\fi
\ifdefined\tablename
  \renewcommand*\tablename{Table}
\else
  \newcommand\tablename{Table}
\fi
}
\@ifpackageloaded{float}{}{\usepackage{float}}
\floatstyle{ruled}
\@ifundefined{c@chapter}{\newfloat{codelisting}{h}{lop}}{\newfloat{codelisting}{h}{lop}[chapter]}
\floatname{codelisting}{Listing}
\newcommand*\listoflistings{\listof{codelisting}{List of Listings}}
\makeatother
\makeatletter
\@ifpackageloaded{caption}{}{\usepackage{caption}}
\@ifpackageloaded{subcaption}{}{\usepackage{subcaption}}
\makeatother
\makeatletter
\@ifpackageloaded{tcolorbox}{}{\usepackage[skins,breakable]{tcolorbox}}
\makeatother
\makeatletter
\@ifundefined{shadecolor}{\definecolor{shadecolor}{rgb}{.97, .97, .97}}
\makeatother
\makeatletter
\makeatother
\makeatletter
\makeatother
\ifLuaTeX
  \usepackage{selnolig}  % disable illegal ligatures
\fi
\IfFileExists{bookmark.sty}{\usepackage{bookmark}}{\usepackage{hyperref}}
\IfFileExists{xurl.sty}{\usepackage{xurl}}{} % add URL line breaks if available
\urlstyle{same} % disable monospaced font for URLs
\hypersetup{
  pdftitle={6.2, 6.4, 6.5 Inference About Means},
  hidelinks,
  pdfcreator={LaTeX via pandoc}}

\title{6.2, 6.4, 6.5 Inference About Means}
\subtitle{Math 261}
\author{}
\date{}

\begin{document}
\frame{\titlepage}
\ifdefined\Shaded\renewenvironment{Shaded}{\begin{tcolorbox}[borderline west={3pt}{0pt}{shadecolor}, breakable, boxrule=0pt, frame hidden, sharp corners, interior hidden, enhanced]}{\end{tcolorbox}}\fi

\begin{frame}{Outline}
\protect\hypertarget{outline}{}
\begin{itemize}
\tightlist
\item
  Formulas for Standard Errors\\
\item
  Introduction to the \(t\) distribution
\item
  \(t\)-based Inference for Means
\end{itemize}
\end{frame}

\begin{frame}{Central Limit Theorem}
\protect\hypertarget{central-limit-theorem}{}
\begin{quote}
For random samples with a \emph{sufficiently large} sample size, the
distribution of sample statistics for a mean or a proportion is
approximately normal.
\end{quote}

\begin{itemize}
\tightlist
\item
  For means, ``sufficiently large'' is often \(n \ge 30\)\\
\item
  If the data are normal, smaller \(n\) will be sufficient\\
\item
  If the data are skewed and/or have outliers, \(n\) may have to be much
  higher than 30
\end{itemize}
\end{frame}

\begin{frame}{Sample Standard Error Formulas}
\protect\hypertarget{sample-standard-error-formulas}{}
\begin{longtable}[]{@{}
  >{\centering\arraybackslash}p{(\columnwidth - 4\tabcolsep) * \real{0.2639}}
  >{\centering\arraybackslash}p{(\columnwidth - 4\tabcolsep) * \real{0.2639}}
  >{\centering\arraybackslash}p{(\columnwidth - 4\tabcolsep) * \real{0.4722}}@{}}
\toprule\noalign{}
\begin{minipage}[b]{\linewidth}\centering
Parameter
\end{minipage} & \begin{minipage}[b]{\linewidth}\centering
Distribution
\end{minipage} & \begin{minipage}[b]{\linewidth}\centering
Standard Error
\end{minipage} \\
\midrule\noalign{}
\endhead
Proportion & Normal & \(\sqrt{\frac{\hat p(1-\hat p)}{n}}\) \\
Difference in Proportions & Normal &
\(\sqrt{\frac{\hat p_1(1-\hat p_1)}{n_1}+\frac{\hat p_2(1-\hat p_2)}{n_2}}\) \\
Mean & \(t\), df\(=n-1\) & \(\sqrt{\frac{s^2}{n}}\) \\
Difference in Means & \(t\), df\(=\min(n_1,n_2)-1\) &
\(\sqrt{\frac{s_1^2}{n_1}+\frac{s_2^2}{n_2}}\) \\
\bottomrule\noalign{}
\end{longtable}
\end{frame}

\begin{frame}{SE of a Mean}
\protect\hypertarget{se-of-a-mean}{}
The standard error for a sample mean can be calculated by

\[
\LARGE 
\mbox{SE}=\frac{\sigma}{\sqrt{n}}
\]
\end{frame}

\begin{frame}{Three Questions}
\protect\hypertarget{three-questions}{}
\begin{figure}

{\centering \includegraphics{Sec6_means_spr24_handout_files/mediabag/plCDRJc3Bbjv.gif}

}

\caption{Three important questions}

\end{figure}
\end{frame}

\begin{frame}{Three Questions (\(\pm 1\))}
\protect\hypertarget{three-questions-pm-1}{}
\begin{itemize}
\item
  What is the standard deviation of the \emph{population}?
\item
  What is the standard deviation of the \emph{sample}?
\item
  What is the standard error of the \emph{sample mean}?
\item
  What is the \emph{estimated} standard error of the \emph{sample mean}?
\end{itemize}
\end{frame}

\begin{frame}{The \(t\)-Distribution}
\protect\hypertarget{the-t-distribution}{}
\begin{itemize}
\item
  For quantitative data, we use a \(t\)-distribution instead of the
  normal distribution
\item
  Reason: Using \(s\) from the sample to estimate \(\sigma\) in the SE
  formula
\item
  The \(t\) distribution is very similar to the standard normal, but
  with slightly thicker tails (to reflect the uncertainty in the sample
  standard deviations)
\item
  Use
  \href{https://www.lock5stat.com/StatKey/theoretical_distribution/theoretical_distribution.html\#t}{StatKey}
  to get \(p\)-values for hypothesis tests and critical values \(t^*\)
  for confidence intervals.
\end{itemize}
\end{frame}

\begin{frame}{Degrees of Freedom}
\protect\hypertarget{degrees-of-freedom}{}
\begin{itemize}
\tightlist
\item
  The \(t\)-distribution is characterized by its \emph{degrees of
  freedom (df)}\\
\item
  Degrees of freedom are based on the sample size

  \begin{itemize}
  \tightlist
  \item
    Single mean: \(df = n – 1\)\\
  \item
    Difference in means: \(df = \min(n_1, n_2) – 1\)
  \item
    ANOVA: \(df = n – K\)
  \end{itemize}
\item
  The higher the degrees of freedom, the closer the \(t\)-distribution
  is to the standard normal.
\end{itemize}
\end{frame}

\begin{frame}{\(t\)-Distribution versus Normal Distribution}
\protect\hypertarget{t-distribution-versus-normal-distribution}{}
\includegraphics{Sec6_means_spr24_handout_files/mediabag/600px-Student_t_pdf..png}
\end{frame}

\begin{frame}{Where was the first \(t\)-Test done?}
\protect\hypertarget{where-was-the-first-t-test-done}{}
\includegraphics{Sec6_means_spr24_handout_files/mediabag/kalamazoo-map-1861-a.jpg}
\end{frame}

\begin{frame}{Case Study: Treatments for Anorexia Nervosa}
\protect\hypertarget{case-study-treatments-for-anorexia-nervosa}{}
\begin{itemize}
\item
  \textbf{Anorexia nervosa} is an eating disorder characterized by
  weight loss (or lack of appropriate weight gain in growing children);
  difficulties maintaining an appropriate body weight for height, age,
  and stature; and, in many individuals, distorted body image.
  (https://www.nationaleatingdisorders.org/anorexia-nervosa)
\item
  Randomized controlled experiment in the UK to assess effectiveness of
  two experimental treatments compared with the established
  \emph{control} treatment (Hand, D. J., Daly, F., McConway, K., Lunn,
  D. and Ostrowski, E. eds (1993) \emph{A Handbook of Small Data Sets.
  Chapman \& Hall}, Data set 285 (p.~229))
\end{itemize}
\end{frame}

\begin{frame}{\(t\)-Based Formulas}
\protect\hypertarget{t-based-formulas}{}
\begin{itemize}
\item
  Confidence interval\\
  \[
  \mbox{sample statistic} \pm t^* \times \mbox{SE}
  \]
\item
  Hypothesis test\\
  \[
   t=\frac{\mbox{sample statistic}-\mbox{null parameter}}{\mbox{SE}}
  \]
\item
  Use the \(t\) distribution for \(p\)-values and critical \(t^*\)
  values.
\end{itemize}
\end{frame}

\begin{frame}{Case Study: Treatments for Anorexia Nervosa}
\protect\hypertarget{case-study-treatments-for-anorexia-nervosa-1}{}
\begin{itemize}
\tightlist
\item
  Today we focus on two groups:

  \begin{itemize}
  \tightlist
  \item
    Control: 26 girls
  \item
    Family therapy: 17 girls
  \end{itemize}
\item
  Variables:

  \begin{itemize}
  \tightlist
  \item
    Group\\
  \item
    Before Weight (\emph{Wtbef})\\
  \item
    After Weight (\emph{Wtaft})\\
  \item
    Gain= After-Before (\emph{Gain})
  \end{itemize}
\end{itemize}
\end{frame}

\begin{frame}{Case Study: Research Questions}
\protect\hypertarget{case-study-research-questions}{}
\begin{enumerate}
\tightlist
\item
  Do girls on family therapy gain weight?

  \begin{itemize}
  \tightlist
  \item
    Hypothesis test
  \end{itemize}
\item
  How much weight do girls on family therapy gain?

  \begin{itemize}
  \tightlist
  \item
    Confidence interval
  \end{itemize}
\item
  Do girls on family therapy gain \textbf{more} weight than girls on
  control therapy?

  \begin{itemize}
  \tightlist
  \item
    Hypothesis test
  \end{itemize}
\item
  \textbf{How much more} weight do girls on family therapy gain than
  girls on control therapy?

  \begin{itemize}
  \tightlist
  \item
    Confidence interval
  \end{itemize}
\end{enumerate}
\end{frame}

\begin{frame}{Matched Pairs}
\protect\hypertarget{matched-pairs}{}
\begin{itemize}
\item
  For a matched pairs experiment, we look at the differences for each
  pair, and do analysis on this one quantitative variable
\item
  Inference for a single mean (mean difference)
\end{itemize}
\end{frame}

\begin{frame}[fragile]{Getting the Data into R}
\protect\hypertarget{getting-the-data-into-r}{}
Use the following command to import the data into the data frame
\texttt{Anorexia\_2samp}.

\begin{Shaded}
\begin{Highlighting}[]
\NormalTok{Anorexia\_2samp }\OtherTok{\textless{}{-}}
  \FunctionTok{read.csv}\NormalTok{(}\StringTok{"http://people.kzoo.edu/enordmoe/math261/data/Anorexia\_2samp.csv"}\NormalTok{)}
\end{Highlighting}
\end{Shaded}
\end{frame}

\begin{frame}{Results for the Family Therapy Girls}
\protect\hypertarget{results-for-the-family-therapy-girls}{}
\begin{itemize}
\item
  The average before weight of family therapy girls was 83.2 pounds.
\item
  The average weight gain for girls in the family therapy group was 7.26
  pounds.
\item
  The standard deviation of these gains was 7.16 pounds.
\item
  The sample size was 17 girls.
\end{itemize}
\end{frame}

\begin{frame}{Q1: Weight gain for girls in Family Therapy: Hypothesis
Test}
\protect\hypertarget{q1-weight-gain-for-girls-in-family-therapy-hypothesis-test}{}
\begin{enumerate}
\item
  State hypotheses
\item
  Check conditions
\item
  Calculate standard error SE
\item
  Calculate \(t\)-statistic
\item
  Compute \(p\)-value
\item
  Interpret in context
\end{enumerate}
\end{frame}

\begin{frame}
\begin{block}{Q1: Weight gain for girls in Family Therapy: Hypothesis
Test Calculations}
\protect\hypertarget{q1-weight-gain-for-girls-in-family-therapy-hypothesis-test-calculations}{}
\end{block}
\end{frame}

\begin{frame}{Q2: Weight gain for girls in Family Therapy: Confidence
Interval}
\protect\hypertarget{q2-weight-gain-for-girls-in-family-therapy-confidence-interval}{}
\begin{enumerate}
\item
  Check conditions
\item
  Find \(t^*\) corresponding to desired level of confidence
\item
  Compute the confidence interval
\item
  Interpret in context
\end{enumerate}
\end{frame}

\begin{frame}
\begin{block}{Q2: Weight gain for girls in Family Therapy: Confidence
Interval Calculations}
\protect\hypertarget{q2-weight-gain-for-girls-in-family-therapy-confidence-interval-calculations}{}
\end{block}
\end{frame}

\begin{frame}{Question 3}
\protect\hypertarget{question-3}{}
\textbf{Hypothesis Test: Do girls on family therapy gain more weight
than girls on control therapy?}

\begin{enumerate}
\item
  State hypotheses

  \begin{itemize}
  \tightlist
  \item
    Two independent samples
  \end{itemize}
\item
  Check conditions
\item
  Calculate standard error SE
\item
  Calculate \(t\)-statistic
\item
  Compute \(p\)-value
\item
  Interpret in context
\end{enumerate}
\end{frame}

\begin{frame}[fragile]{Summary Results for Family Therapy and Control
Groups}
\protect\hypertarget{summary-results-for-family-therapy-and-control-groups}{}
\begin{Shaded}
\begin{Highlighting}[]
\FunctionTok{favstats}\NormalTok{(Gain }\SpecialCharTok{\textasciitilde{}}\NormalTok{ Group, }\AttributeTok{data =}\NormalTok{ Anorexia\_2samp)}
\end{Highlighting}
\end{Shaded}

\begin{verbatim}
    Group   min   Q1 median   Q3  max      mean       sd  n missing
1 Control -12.2 -7.0  -0.35  3.6 15.9 -0.450000 7.988705 26       0
2  Family  -5.3  3.9   9.00 11.4 21.5  7.264706 7.157421 17       0
\end{verbatim}
\end{frame}

\begin{frame}{Question 3}
\protect\hypertarget{question-3-1}{}
\textbf{Do girls on family therapy gain more weight than girls on
control therapy? Hypothesis Test Calculations}
\end{frame}

\begin{frame}{Question 4}
\protect\hypertarget{question-4}{}
\textbf{How much more weight do girls on family therapy gain than girls
on control therapy? Confidence interval}

\begin{enumerate}
\item
  Check conditions
\item
  Find \(t^*\) corresponding to desired level of confidence
\item
  Compute the confidence interval
\item
  Interpret in context
\end{enumerate}
\end{frame}

\begin{frame}
\begin{block}{Question 4}
\protect\hypertarget{question-4-1}{}
\textbf{How much more weight do girls on family therapy gain than girls
on control therapy? Confidence interval calculations}
\end{block}
\end{frame}

\begin{frame}{Inference formulas for means}
\protect\hypertarget{inference-formulas-for-means}{}
\begin{longtable}[]{@{}
  >{\centering\arraybackslash}p{(\columnwidth - 4\tabcolsep) * \real{0.2500}}
  >{\centering\arraybackslash}p{(\columnwidth - 4\tabcolsep) * \real{0.3750}}
  >{\centering\arraybackslash}p{(\columnwidth - 4\tabcolsep) * \real{0.3750}}@{}}
\toprule\noalign{}
\begin{minipage}[b]{\linewidth}\centering
Parameter of Interest
\end{minipage} & \begin{minipage}[b]{\linewidth}\centering
Confidence Interval
\end{minipage} & \begin{minipage}[b]{\linewidth}\centering
Test of Significance
\end{minipage} \\
\midrule\noalign{}
\endhead
\(\mu\) & \(\bar x \pm t^* \frac{s}{\sqrt{n}}\) &
\(t=\frac{\bar x-\mu_0}{s/\sqrt{n}}\) \\
\(\mu_1-\mu_2\) &
\((\bar x_1-\bar x_2) \pm t^* \sqrt{\frac{s_1^2}{n_1}+\frac{s_2^2}{n_2}}\)
&
\(t=\frac{(\bar x_1-\bar x_2)}{\sqrt{\frac{s_1^2}{n_1}+\frac{2_2^2}{n_2}}}\) \\
\bottomrule\noalign{}
\end{longtable}
\end{frame}

\hypertarget{a-few-words-about-conditions}{%
\section{A Few Words about
Conditions}\label{a-few-words-about-conditions}}

\begin{frame}{Normality Conditions}
\protect\hypertarget{normality-conditions}{}
\begin{itemize}
\item
  Using the t-distribution requires that the data comes from a
  \emph{normal distribution}
\item
  Note: this assumption is about the population data, \emph{not} the
  distribution of the statistic.
\item
  For large sample sizes we do not need to worry about this, because
  \(s\) will be a very good estimate of \(\sigma\) and \(t\) will be
  very close to \(N(0,1)\).
\end{itemize}
\end{frame}

\begin{frame}{Small Samples}
\protect\hypertarget{small-samples}{}
\begin{itemize}
\item
  For \textbf{small sample sizes} (\(n<30\)), we can only use the
  \(t\)-distribution if the distribution of the data is approximately
  \textbf{normal}.

  \begin{itemize}
  \tightlist
  \item
    Problem: Hard to assess normality for small samples.
  \end{itemize}
\end{itemize}

\pause

\begin{itemize}
\item
  If sample sizes are \textbf{small}, only use the \(t\)-distribution if
  the data look reasonably symmetric and do not have any extreme
  outliers.

  \begin{itemize}
  \item
    Even then, remember that the \(t\) is just an approximation!
  \item
    In practice, if sample sizes are small, use \textbf{simulation
    methods} (bootstrapping and randomization).
  \end{itemize}
\end{itemize}
\end{frame}

\begin{frame}{Summary}
\protect\hypertarget{summary}{}
\begin{itemize}
\tightlist
\item
  \textbf{Understanding the \(t\)-distribution}

  \begin{itemize}
  \tightlist
  \item
    Accounts for additional uncertainty in sample estimates
  \item
    Thicker tails compared to the normal distribution
  \end{itemize}
\item
  \textbf{Inference procedures}

  \begin{itemize}
  \tightlist
  \item
    Use \(t\)-based confidence intervals and hypothesis tests for means
  \item
    Consider degrees of freedom
  \item
    Ensure conditions for using the \(t\)-distribution are met
  \end{itemize}
\item
  \textbf{Practical application}

  \begin{itemize}
  \tightlist
  \item
    Analyze weight gain data from anorexia nervosa treatment study
  \item
    Apply hypothesis tests and confidence intervals
  \item
    Draw conclusions about treatment effectiveness
  \end{itemize}
\end{itemize}
\end{frame}



\end{document}
